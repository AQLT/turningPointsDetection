\thispagestyle{fancy}

\hypertarget{ruxe9sumuxe9s}{%
\section*{Résumés}\label{ruxe9sumuxe9s}}
\addcontentsline{toc}{section}{Résumés}

\begin{abstract}
La crise du COVID-19 met en évidence que l'analyse des cycles
économiques, et en particulier la détection précoce des points de
retournement, est un sujet majeur dans l'analyse de la conjoncture. Les
moyennes mobiles, ou les filtres linéaires, sont omniprésents dans les
méthodes d'extraction du cycle économique. Au centre de la série, des
filtres symétriques sont appliqués. Cependant, en raison du manque
d'observations futures, les estimations en temps réel doivent s'appuyer
sur des moyennes mobiles asymétriques. Les moyennes mobiles asymétriques
classiques minimisent les erreurs de révision mais introduisent des
retards dans la détection des points de retournement (déphasage).

La construction de moyennes mobiles asymétriques performantes en termes
de fidélité (de préservation du signal), de révision, de lissage et de
déphasage est un sujet de recherche toujours ouvert. Ce rapport décrit
et compare des approches récentes pour construire des filtres
asymétriques : filtres polynomiaux locaux, méthodes basées sur une
optimisation des propriétés des filtres et filtres basés sur les espaces
de Hilbert à noyau reproduisant (RKHS). Il décrit également comment les
filtres polynomiaux locaux peuvent être étendus pour inclure un critère
de temporalité afin de minimiser le déphasage. Toutes ces méthodes
peuvent se voir comme un cas particulier d'une théorie générale de
construction des filtres et ont été comparées en les intégrant dans la
méthode de désaisonnalisation X-13ARIMA.

Ce rapport montre notamment que contraindre les moyennes mobiles
asymétriques à ne conserver que les constantes (et pas nécessairement
polynomiales) réduit à la fois l'erreur de révision et le déphasage dans
la détection de points de retournement. Les futures études sur le sujet
peuvent donc se concentrer sur ces filtres. Ce rapport met également en
évidence le lien entre désaisonnalisation et extraction de
tendance-cycle. Les deux ne peuvent s'étudier de manière indépendante et
négliger la méthode de désaisonnalisation peut conduire à des
estimations de la tendance-cycle biaisées par la présence de points
atypiques mais aussi par l'introduction de faux points de retournement.

Toutes les méthodes décrites sont implémentées dans le package
\faIcon{r-project} \texttt{rjdfilters}
(\url{https://github.com/palatej/rjdfilters}) et tous les résultats
peuvent facilement être reproduits. Les codes utilisés, ainsi qu'une
version web de ce rapport, sont disponibles sur
\url{https://github.com/AQLT/Stage_3A}.

\end{abstract}

\renewcommand{\abstractname}{Abstract}

\begin{abstract}
The COVID-19 crisis highlights that business cycle analysis, and in
particular the early detection of turning points, is a major topic in
the analysis of economic outlook. In the business cycle analysis,
estimates are usually derived from moving average (also called linear
filters) techniques. In the center of the series, symmetric filters are
applied. However, due to the lack of future observations, real-time
estimates must rely on asymmetric moving averages. Classic asymmetric
moving averages minimize revisions errors but introduce delays in the
detecting turning points.

Construction of good asymmetric filters, in terms of fidelity,
revisions, smoothness and timeliness, is still an open topic. This paper
describes and compares different approaches to build asymmetric filters:
local polynomials filters, methods based on an optimization of filters'
properties (Fidelity-Smoothness-Timeliness, FST, approach and a
data-dependent filter) and filters based on Reproducing Kernel Hilbert
Space. It also describes how local polynomials filters can be extended
to include a timeliness criterion to minimize phase shift. All these
methods can be seen as a special case of a general unifying framework to
derive linear filters, and have been compared by integrating them into
the X-13ARIMA seasonal adjustment method.

This paper shows that constraining asymmetric filters to preserve
constant trends (and not necessarily polynomial ones) reduce revision
error and time lag. Therefore, future studies on the subject can focus
on these filters. This report also highlights the link between seasonal
adjustment method and trend-cycle extraction methods. Both cannot be
studied independently and neglecting the seasonal adjustment method can
lead to trend-cycle estimates biased by the presence of outliers but
also by the introduction of false turning points.

All the methods are implemented in the \faIcon{r-project} package
\texttt{rjdfilters} (\url{https://github.com/palatej/rjdfilters}) and
the results can be easily reproduced. The programs used, and a web
version of this report, are available at
\url{https://github.com/AQLT/Stage_3A}.

\end{abstract}

\newpage
